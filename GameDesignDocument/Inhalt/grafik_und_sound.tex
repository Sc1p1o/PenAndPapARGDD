In diesem Abschnitt werden die zentralen Grafiken vorgestellt, die im Spiel verwendet werden, sowie ihre Bedeutung
und Platzierung im Design.

\subsection{Grafiken für Spielzustände (Conditions)}
Die folgenden Grafiken stellen verschiedene Bedingungen oder Zustände dar, die auf Charaktere im Spiel angewandt werden
können.\ Jeder Zustand wird durch eine Grafik illustriert, um die Benutzerfreundlichkeit und Verständlichkeit zu erhöhen.

Dieser Abschnitt bietet eine Übersicht über wichtige Grafiken, die im Spiel verwendet werden.
Für jede Grafik wird das entsprechende Bild angezeigt, zusammen mit einer kurzen Beschreibung.

Die folgenden Tabellen zeigen die einzelnen Spielzustände, illustriert durch ihre jeweilige Grafik und eine Beschreibung:
\newline
\newline
\renewcommand{\arraystretch}{1.5} % Zeilenabstand in der Tabelle erhöhen
\begin{longtable}{|p{4cm}|p{10cm}|}
\hline

\includegraphics[width=2cm]{../images/Conditions/blinded.png} &
\textbf{Blinded:} Der Zustand eines geblendeten Charakters, wodurch dessen Sicht stark eingeschränkt wird.
Dies kann Einfluss auf die Aktionen und Wahrnehmung haben.
\hline

\includegraphics[width=2cm]{../images/Conditions/poisoned.png} &
\textbf{Poisoned:} Dieser Zustand stellt einen vergifteten Charakter dar.
Der Charakter erleidet schrittweise Schaden oder wird in seinen Werten geschwächt.
\hline

\includegraphics[width=2cm]{../images/Conditions/restrained.png} &
\textbf{Restrained:} Zeigt einen Zustand begrenzter Bewegung, in dem der Charakter gefesselt oder anderweitig
unbeweglich gemacht wurde.\ Angriffe gegen ihn haben häufig Vorteile.
\hline

\includegraphics[width=2cm]{../images/Conditions/exhaustion.png} &
\textbf{Exhaustion:} Die Erschöpfung eines Charakters, der durch Beeinträchtigung mehrerer Werte schrittweise
geschwächt wird (z.B.\ Geschwindigkeit oder Lebenspunkte, Attribute, Skills).
\hline

\includegraphics[width=2cm]{../images/Conditions/frightened.png} & \textbf{Frightened:} Dieser Zustand symbolisiert die Angst eines Charakters, wodurch seine Bewegungsfreiheit und
sein Verhalten eingeschränkt werden können.


\includegraphics[width=2cm]{../images/Conditions/grappled.png} &
\textbf{Grappled:} Der Zustand zeigt einen festgehaltenen Charakter, der sich nicht frei bewegen kann, bis er sich
befreit. \\
\hline

\includegraphics[width=2cm]{../images/Conditions/charmed.png} &
\textbf{Charmed:} Zustände unter dem Einfluss äußerer Kräfte (z.B.\ Verzauberung), wobei der Charakter möglicherweise
nicht frei agieren kann. \\
\hline

\includegraphics[width=2cm]{../images/Conditions/paralyzed.png} &
\textbf{Paralyzed:} Vollkommene Bewegungsunfähigkeit, wobei der Charakter keinen Widerstand leisten kann.
\hline

\includegraphics[width=2cm]{../images/Conditions/invisible.png} &
\textbf{Invisible:} Dieser Zustand macht den Charakter für Gegner unsichtbar, wodurch Nah- oder Fernangriffe erschwert
werden.
\hline

\includegraphics[width=2cm]{../images/Conditions/deafened.png} &
\textbf{Deafened:} Darstellung der Taubheit eines Charakters, die es schwierig macht, akustische Signale wahrzunehmen.
\hline

\includegraphics[width=2cm]{../images/Conditions/unconscious.png} &
\textbf{Unconscious:} Ein bewusstloser Charakter ist kampfunfähig und anfällig für Angriffe.
\hline

\includegraphics[width=2cm]{../images/Conditions/incapacitated.png} &
\textbf{Incapacitated:} Zeigt einen Zustand der völligen Handlungsfähigkeit, bei dem keine Aktionen ausgeführt werden
können.

\includegraphics[width=2cm]{../images/Conditions/prone.png} &
\textbf{Prone:} Der liegende Zustand eines Charakters, der seine Manövrierfähigkeit und seinen Verteidigungswert
verringert.
\hline

\includegraphics[width=2cm]{../images/Conditions/stunned.png} &
\textbf{Stunned:} Ein betäubter Charakter ist nicht in der Lage, sich zu bewegen oder auf Aktionen anderer zu reagieren.
\hline

\includegraphics[width=2cm]{../images/Conditions/initiative.png} &
\textbf{Initiative:} Zeigt die Reihenfolge und Reaktionszeit von Akteuren im Spiel an.
\hline

\end{longtable}




%    Spieler ansicht:
%    - Herz, welches die aktuellen HP anzeigt
%    - XP Balken
%    - Schild, das den aktuellen Verteidigungswert anzeigt
%    - aufzählung der jeweiligen Stärke, Intelligenz, Konstitution, Weisheit, Charisma
%    - evtl Minibild von kontrollierten NPCs
%    - kurzbild von Statuseffekten
%
%    Detaillierte Ansicht:
%    - genaue Charaktere Statistiken; Charaktersheet
%    - Informationen über Inventar (Inhalt, Platz, usw.)
%    - detaillierte Statuseffekte
%    - kontrollierte NPCs in Namensliste
%
%    \section{Soundeffects}\label{sec:soundeffects}
%    - Link zu Spotify/Youtube um Musik für Atmosphäre zu steuern
%    - evtl Soundeffekte für bestimmte Spiel Events, Explosionen, Angriffe