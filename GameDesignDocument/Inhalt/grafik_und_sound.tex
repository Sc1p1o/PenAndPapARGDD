In diesem Abschnitt werden die zentralen Grafiken vorgestellt, die im Spiel verwendet werden, sowie ihre Bedeutung
und Platzierung im Design.

\subsection{Attribute}\label{subsec:atribute_graphics}
Zu aller erst werden die einzelnen Attribute eines Charakters grafisch angezeigt durch ein dazu passendes Icon.
Diese Attribute beeinflussen einen Charakter und seine einzelnen Skills direkt oder indirekt durch Rettungswürfe oder
Rollenspielaspekte.
\newline
\newline

\renewcommand{\arraystretch}{1.5}
\begin{longtable}
    \includegraphics[width=2cm]{../images/Icon/strength} &
    \textbf{Stärke:} Stärke steht für physische Kraft und beeinflusst die Traglast, Nahkampfangriffe und andere
    physische Aktionen.

    \hline

    \includegraphics[width=2cm]{../images/Icon/dexterity} &
    \textbf{Geschicklichkeit:} Die Geschicklichkeit bestimmt die Beweglichkeit und Reflexe eines Charakters.
    Sie beeinflusst die Charakterinitiative und viele Fähigkeiten die zur Bewegung notwendig sind.
    \hline

    \includegraphics[width=2cm]{../images/Icon/constitution} &
    \textbf{Konstitution:} Die Konstitution steht für die Gesundheit eines Charakters und bestimmt essenziel mit wie gut
    man Giften und einflüssen wie Kälte und Hitze widerstehen kann.
    Auch beeinflusst sie die Lebenspunkte eines Charakters.
    \hline

    \includegraphics[width=2cm]{../images/Icon/intelligence} &
    \textbf{Intelligenz:} Die Intelligenz repräsentiert den Wissensstand und das logische Denken eines Charakters.
    Sie beeinflusst lern bezogene Fähigkeiten und Zauber-Fähigkeiten bestimmter Klassen.
    \hline

    \includegraphics[width=2cm]{../images/Icon/wisdom} &
    \textbf{Weisheit:} Die Weisheit repräsentiert Wahrnehmung und Intuition.\ Sie beeinflusst Fähigkeiten wie Medizin,
    Überleben und bestimmte Zauber.
    \hline

    \includegraphics[width=2cm]{../images/Icon/charisma} &
    \textbf{Charisma:} Dieses Attribut repräsentiert die Ausstrahlung und Überzeugungskraft eines Charakters.
    Es beeinflusst soziale Interaktionen und bestimmte Fähigkeiten wie Verhandeln oder Täuschen.
    \hline

\end{longtable}


\subsection{Grafiken für Spielzustände (Conditions)}\label{subsec:condition_graphics}
Die folgenden Grafiken stellen verschiedene Bedingungen oder Zustände dar, die auf Charaktere im Spiel angewandt
werden können.\ Jeder Zustand wird durch eine Grafik illustriert, um die Benutzerfreundlichkeit und Verständlichkeit zu
erhöhen.

Dabei gibt es im Spiel folgende Zustände die ein Spieler erhalten kann:

\newline
\newline
\renewcommand{\arraystretch}{1.5}
\begin{longtable}
    \hline

    \includegraphics[width=2cm]{../images/Conditions/blinded} &
    \textbf{Blinded:} Der Zustand eines geblendeten Charakters, wodurch dessen Sicht stark eingeschränkt wird.
    Dies kann Einfluss auf die Aktionen und Wahrnehmung haben.
    \hline

    \includegraphics[width=2cm]{../images/Conditions/poisoned} &
    \textbf{Poisoned:} Dieser Zustand stellt einen vergifteten Charakter dar.
    Der Charakter erleidet schrittweise Schaden oder wird in seinen Werten geschwächt.
    \hline

    \includegraphics[width=2cm]{../images/Conditions/restrained} &
    \textbf{Restrained:} Zeigt einen Zustand begrenzter Bewegung, in dem der Charakter gefesselt oder anderweitig
    unbeweglich gemacht wurde.\ Angriffe gegen ihn haben häufig Vorteile.
    \hline

    \includegraphics[width=2cm]{../images/Conditions/exhaustion} &
    \textbf{Exhaustion:} Die Erschöpfung eines Charakters, der durch Beeinträchtigung mehrerer Werte schrittweise
    geschwächt wird (z.B.\ Geschwindigkeit oder Lebenspunkte, Attribute, Skills).
    \hline

    \includegraphics[width=2cm]{../images/Conditions/frightened} & \textbf{Frightened:} Dieser Zustand symbolisiert
    die Angst eines Charakters, wodurch seine Bewegungsfreiheit und sein Verhalten eingeschränkt werden können.


    \includegraphics[width=2cm]{../images/Conditions/grappled} &
    \textbf{Grappled:} Der Zustand zeigt einen festgehaltenen Charakter, der sich nicht frei bewegen kann, bis er sich
    befreit. \\
    \hline

    \includegraphics[width=2cm]{../images/Conditions/charmed} &
    \textbf{Charmed:} Zustände unter dem Einfluss äußerer Kräfte (z.B.\ Verzauberung), wobei der Charakter möglicherweise
    nicht frei agieren kann. \\
    \hline

    \includegraphics[width=2cm]{../images/Conditions/paralyzed} &
    \textbf{Paralyzed:} Vollkommene Bewegungsunfähigkeit, wobei der Charakter keinen Widerstand leisten kann.
    \hline

    \includegraphics[width=2cm]{../images/Conditions/invisible} &
    \textbf{Invisible:} Dieser Zustand macht den Charakter für Gegner unsichtbar, wodurch Nah- oder Fernangriffe erschwert
    werden.
    \hline

    \includegraphics[width=2cm]{../images/Conditions/deafened} &
    \textbf{Deafened:} Darstellung der Taubheit eines Charakters, die es schwierig macht, akustische Signale wahrzunehmen.
    \hline

    \includegraphics[width=2cm]{../images/Conditions/unconscious} &
    \textbf{Unconscious:} Ein bewusstloser Charakter ist kampfunfähig und anfällig für Angriffe.
    \hline

    \includegraphics[width=2cm]{../images/Conditions/incapacitated} &
    \textbf{Incapacitated:} Zeigt einen Zustand der völligen Handlungsfähigkeit, bei dem keine Aktionen ausgeführt werden
    können.

    \includegraphics[width=2cm]{../images/Conditions/prone} &
    \textbf{Prone:} Der liegende Zustand eines Charakters, der seine Manövrierfähigkeit und seinen Verteidigungswert
    verringert.
    \hline

    \includegraphics[width=2cm]{../images/Conditions/stunned} &
    \textbf{Stunned:} Ein betäubter Charakter ist nicht in der Lage, sich zu bewegen oder auf Aktionen anderer zu reagieren.
    \hline

\end{longtable}

\subsection{Weitere Icons}\label{subsec:other_graphics}
Zudem gibt es noch einzelne andere Grafiken, die wir integrieren wollen, um immer mehr Text zu ersetzen und bildlich
mit den Spielern zu sprechen.

\newline
\newline

\renewcommand{\arraystretch}{1.5}
\begin{longtable}

    \includegraphics[width=2cm]{../images/Icon/ac} &
    \textbf{Rüstungsklasse:} Die Rüstungsklasse zeigt die Verteidigungsfähigkeit eines Charakters an.
    Je höher der Wert, desto schwieriger ist es, den Charakter zu treffen.
    \hline

    \includegraphics[width=2cm]{../images/Icon/expertise} &
    \textbf{Expertise:} Ein Status den man durch Klassen einigen Skills zuweisen kann um den Proficiency Bonus gleich
    zweimal statt nur einmal dazu addieren kann.
    \hline

    \includegraphics[width=2cm]{../images/Icon/full} &
    \textbf{Lebenspunkte:} Das Herz wird prozentual gefüllt abhängig vom gesamten HP Anzahl zu der aktuellen HP Anzahl.
    \hline

    \includegraphics[width=2cm]{../images/Conditions/initiative} &
    \textbf{Initiative:} Zeigt die Reihenfolge und Reaktionszeit von Akteuren im Spiel an.
    \hline

\end{longtable}