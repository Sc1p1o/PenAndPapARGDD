\usepackage{eurosym}Das gesamte Projekt hat nicht das große Geld oder große Gewinne als Plan oder Konzept.
Es sollte stets beachtet werden.\ Somit ist auch die Arbeit am Projekt aktuell eine rein ehrenamtliche Beschäftigung.
Allerdings werden gewisse Kosten entstehen zum Bereitstellen der Datenbank und auch für Lizenz gebühren durch die
Nutzung von Unity.

Da aber sicher nicht jeder diesen Service zwingend braucht, lohnt es sich eventuell, eine Grundversion anzubieten,
mit einer kleinen Anleitung zum Erstellen einer eigenen Datenbank vor Ort, zum eigenen Gebrauch mit Freunden, welche
kostenlos sein kann.\newline
Zusätzlich kann man den Zugriff auf einen globalen Server abonnieren und für 2 € pro Monat unsere Kosten zum
instandhalten des Servers decken.\ Weiter kann man einen Rabatt anbieten, wenn eine ganze Gruppe zusammen den
Server nutzen will.\ Man könnte so zum Beispiel für 5 € einen Zugang für 6 Personen anbieten.
Der Rabatt ist möglich weil, a: die meisten Gruppen meist aus 4 bis 5 Personen bestehen, nur manchmal auch eine 6 Person
dazu kommt und b: die Kosten der Wartung nicht gleich zur Spielerzahl steigen, sondern weitaus langsamer ansteigen.

Die zweite Finanzierung wird dazu genutzt die App zu verbessern und den Entwicklern für ihre Arbeit Anerkennung zu
geben.\ So sollen Spieler die Möglichkeit bekommen Spenden an die Entwickler zu senden.
Hierbei sollen sie auch die Möglichkeit haben eine Einzelspende zu geben oder eine wiederkehrende Spende zu geben.
Bei wiederkehrende Spenden, sollen Spieler, nochmals unsere Dankbarkeit erhalten und ab
einem monatlichen Betrag von über 2 € als Spende, ebenfalls einen Datenbankzugang erhalten für die Dauer
der Spende.