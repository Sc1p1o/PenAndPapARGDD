\subsection{Spaß für Jung und Alt}\label{subsec:fun_for_all_ages}
Die Hauptzielgruppe unseres Spiels wird alterlich sehr unterschiedlich sein.\ Daher ist es wichtig eine intuitive
Steuerung für Personen jeden Alters zu nutzen.\ Damit die Steuerung von allen möglichen Menschen verstanden wird, auch
Eine UI die zu viel Text enthält wird vielleicht die Millenials nicht stören, da diese solche UIs auch aus den Zeiten
der 90er und 2000er Jahre in Internet Seiten kennen.
Doch auch diese und spätestens die Ersten der GenZ sehen eine Intuitive UI ohne viel Text und nach gewohnten Mustern.
Somit ist es zwingend erforderlich Bilder sprechen zu lassen statt Text.\ Komplett auf Text und Zahlen verzichten wird
nicht möglich sein, da Pen and Paper Systeme enorm Groß sind und man kaum alles mit Bildern eindeutig betiteln kann.
Wie zum Beispiel die Skills.\ Schleichen und arkanes Wissen mögen zwar gut visualisierbar sein.
Schwieriger wird es aber bei Akrobatischen und athletischen Fähigkeiten, diese so zu trennen, dass sie eindeutig sind
und das Icon nicht zu gross ist.

\subsection{Wer sind unsere Spieler?}\label{subsec:who_is_your_player}
Unsere Spieler werden wohl vor allem eines sein: Kreativ!\ Wer Pen and Paper Systeme wie Dungeons and Dragons spielt
ist Text gewöhnt, er ist motiviert ein komplexes Regelsystem zu lernen, weil es ihm eine fantastische Welt eröffnet,
Geschichten erzählen und spielen lässt.\ Eigene und fremde Ideen entdecken lässt und in fremde Welten tauchen lässt.
Ohne Kreativität wäre Dungeons and Dragons nur eine aneinander Reihung von Würfel Würfen.
Daher sollten wir beim Entwickeln unserer Applikation darauf achten, diesen kreativen Freigeistern auch nicht die
Möglichkeit wegnehmen, ihre kreativität in ihrer Kamapgne auszuleben und Regeln und Welten von bereits vorhandenen
Systemen anzupassen und an die eigenen Bedürfnisse zu verbessern.\ Denn von dieser Freiheit lebt ein Pen and Paper
und so auch unser Spiel.