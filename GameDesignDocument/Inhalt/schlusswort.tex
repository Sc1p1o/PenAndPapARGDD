\subsection{Der aktuelle Stand des Projekts}\label{subsec:current_state}
Der aktuelle Stand des Projektes ist in etwa kurz vor Abschluss des Abschnitts \textit{Der Mitspieler} im Punkt
\textit{Entwicklung} zu verordnen.\ Vieles der dort genannten Dinge sind bereits integriert und benötigen nur noch einen
gewissen Feinschliff.\ Nach diesen Änderungen sollte das Projekt in die nächste Phase gehen können.
Der kritischste noch fehlende Punkt ist das Verknüpfen, der bereits erstellten und im Projekt eingefügten Icons und
Grafiken, mit den GameObjects.\ Dies ist leider aus Zeitgründen nicht mehr gelungen, da die Icons erst kurz vor Ende des
Projekts hinzugefügt wurden.

\subsection{Die Zukunft}\label{subsec:back_to_the_future}
Stand jetzt wird das Projekt wahrscheinlich zu großem Teil neu aufgesetzt, da keiner im Team zu Hause eine HoloLens
oder andere AR Brillen hat.\ Allerdings ist eine portierung für PC und Handheld AR Denkbar und wird wohl weiter
in diese Richtung entwickelt.
Es ist allerdings fraglich wer nach dem Semester noch dabei mithilft und ob das Projekt nicht erstmal pausiert wird aus
zeitlichen Gründen der Teammitglieder.
Alternativ kann eine Basis für weitere AR/VR Semester geschaffen werden, sodass andere Studenten dieses Projekt nicht
wieder von vorne beginnen müssen, sondern auf das bisher geschaffene aufbauen können und dieses Projekt so erweitern
können.\ Im GitHub wird dafür eine Kontaktmöglichkeit eingerichtet welche allerdings wohl erst nach dem 15.03.2025
dort zu finden sein wird.