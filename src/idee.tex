Die Idee hinter dem Spiel \textit{Pen\&PapAR} ist es, den seit Serien wie
\textit{Stranger Things} oder \textit{The Big Bang Theorie} immer beliebteren und
spätestens mit dem Hype um das Spiel \textit{Baldur's Gate 3} und dem Film
\textit{Dungeons und Dragons - Ehre unter Dieben} auch im Mainstream angekommenen
Pen\&Paper Klassiker \textit{Dungeons and Dragons} so wie weitere Pen\&Paper Systeme nun
auch in die heutige Zeit zu holen.\newblock

Dabei ist es uns wichtig das Spiel nicht komplett in die digitale Welt zu holen, sondern
den besten Part, als soziales Zusammenkommen möglichst beizubehalten.
Genauso wichtig ist es den Spielern nichts Digitales aufzuzwingen, dafür müssen
die Spieler die gleichen Freiheiten haben, wie bei einem klassischen Spiel.
Das bedeutet, dass das Charakter-Sheet des Spielers sowohl digital 100\% editierbar ist,
als auch, dass Spieler ihren Grad an digitalisierung frei entscheiden können müssen und
alle potenziell analog ausgeführten Aktionen von Spielern selbst digital nachgetragen werden
können müssen, falls nötig bzw.\ gewünscht.\newblock

Der Vorteil einer digitalisierten Version ist aber vor allem, der Platz, dies sollte auch
Hauptaugenmerk unseres Spiels sein.
Charakter-Sheets sind oft viel zu klein um seinen ganzen Loot und all die mächtigen Angriffe,
Zauber und Fähigkeiten niederzuschreiben, die man im Laufe seiner Kampagne gesammelt hatte.
Dazu sind diese meist mehr als eine Seite lang und sobald man ein Haustier oder einen Begleiter
hat, kann das schnell in einem großen Papier-Chaos enden, grade in der Rolle als Spielleiter
hat man schnell viele Bücher, Regelwerke, und NPC-Sheets auf dem Tisch liegen.\newblock

Mit unserem Spiel ist es einfach und übersichtlich all diese Dinge im Blick zu behalten und
dabei den Tisch freizuhalten.
Mit einem Klick kann man eine Liste aller kontrollierter NPCs
ansehen und deren Details nachschauen oder Informationen zu einem bestimmten Regelwerk oder Notizen
nachzuschauen.
Somit hat man viel mehr Platz am Tisch für die Snacks, Würfelwürfe und die sorgfältig bemalten
Minis.\newblock

Es gibt oft zwar bereits digitale Lösungen, diese beschränken sich aber oft auf Laptops
oder sind nur unter einem kostenpflichtigen Abo abrufbar.
Außerdem ist es größtenteils schwer bei den offiziellen Anbietern fremde sogenannte \textit{Homebrew}-Mechaniken
und Items hinzuzufügen.
Dies soll bei unserem Spiel soweit möglich vereinfacht werden, sodass unsere Spieler nicht erst kompliziert
ein Item konfigurieren müssen und es einfügen müssen, damit es verwendbar wird.
Spieler sollen frei entscheiden können, ob sie die Fähigkeiten eines solchen Gegenstands einfach
in der Beschreibung erläutern oder sich die Mühe machen wollen, die Mechanik vollständig in das
Spiel zu integrieren.
Beide wege sollen dabei gleichermaßen intuitiv sein und die Funktion der Waffe ähnlich
übersichtlich darstellen.\newblock

Wenn all diese Aspekte fertiggestellt sind, ist das nächste Ziel nach dem Komfort und Platzsparen, die Immersion des
Spiels zu erhöhen.\newline
Der erste Part hierbei wird es sein, eine 3D-Karte zu erstellen, welche das aktuelle Schlachtfeld, den Dungeon oder
das aktuelle Areal in 3D darstellt und den Spielleiter interaktiv Punkte auswählen lässt, um Spieler so auf eine
Regionskarte der eine genauere Karte öffnet, auf der die Spieler danach ihre Minis bewegen können.
Dies setzt natürlich für alles eine solche Karte in digitaler Form voraus.
Sollte der Spielleiter keine solche Karte eingestellt haben, kann er dennoch auch über das Menü sofort
eine Kampfkarte seiner Wahl öffnen.
Dieses Feature soll allerdings, für den Fall, dass die Spieler eine selbst gemalte Karte bevorzugen, auch
ausschaltbar sein.
Oberste Priorität ist der Komfort der Spieler ihr Spiel nur so weit zu digitalisieren, wie sie es sich wünschen.
Unser Spiel soll sie dabei so wenig wie möglich zwingen, Elemente des Pen\&Papers zu digitalisieren.\n