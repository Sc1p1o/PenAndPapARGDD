\subsection{Core Game Loop}\label{subsec:chapter_gameloop}

Beim Start lädt das Spiel automatisch den Character aus der Datenbank und füllt die verschiedenen Charaktereigenschaften
mit den entsprechenden Informationen aus.\ Dann können die Spieler auch bereits loslegen und sind spielbereit.\ Durch
Antippen der einzelnen Menüs können sie die kleinen Icons erweitern, um ausführlichere Charaktereigenschaften zu erhalten
und abhängig davon in welchem Layer sie sich befinden, können sie dann auch mehr Eigenschaften bearbeiten.
Auch können sie ihr Inventar verwalten und sich eigene Notizen machen, indem sie auf die jeweils anderen dazugehörigen
Menüs tippen.\ Wenn die Spieler ihre Menüs erstmal aus dem Weg haben wollen, können sie diese durch gedrückt halten auf
das Icon im ersten Layer in 3D Objekte verwandeln und dann im Raum abstellen, der ihnen besser passt.

Auch können sie mit einer eigenen API ihren \textit{DnDBeyond} Charakter verwenden.\ Da \textit{DnDBeyond} nach unserem
Wissensstand aktuell allerdings keine eigene öffentlich zugängliche API zur Verfügung stellt, muss der Charakter auf
\textit{DnDBeyond} öffentlich einsehbar sein.\ Die Datenbank erlaubt dabei auch das Speichern mehrerer Charaktere, da
diese automatisch mit einer ID versehen werden.\ Der Index reicht für 9.999 Charaktere pro Spieler.

Aktionen, Zauber und Feats müssen aktuell allerdings noch in das Feld eingefügt werden.
\newblock