%! Author = mmiller, clenk
%! Date = 10.12.24Initial commit

% Preamble
\documentclass[11pt]{article}

% Packages
\usepackage{amsmath}
\usepackage{titling}
\usepackage{tocloft}


% Document
\begin{document}

    \title{Pen and PapAR}
    \author{Cassandra Lenk und Martin Miller}

    \maketitle
    \newpage

    \tableofcontents
    \newpage

    \section{Idee}
    - Digitalisierung von DnD und ähnlichen Pen&Paper spielen
    - Platz sparen beim Spielen, dadurch das alles digitalisiert ist
    - mehr Platz für snacks
    - immersivere spielerfahrung durch 3D Karten

    \section{Gameplay Loop}
    - nicht direkt vorhanden
    - session start ist beginn einer art loop
    - session ende beendet die loop

    \section{Controls}
    - Steuerung durch Handbewegung und tippen mit Fingern auf 3D projizierte Items
    - Klick auf Spieler öffnet detaillierte Übersicht über dessen Charakter
    - Finger auseinander ziehen/zusammen ziehen bei Map, zoomt rein oder raus
    - anklicken des eigenen Menüs gibt eigenen Charakter detailliert an
    - anklicken kontrollierter NPCs gibt genauere Informationen zu diesen
    - Map Auswahl geschieht über Menü

    \section{The Player}
    DM


    Regular PLayer


    \section{Systems}
    - Hololens für AR
    - Server für lokalen Mehrspielermodus
    - JSON Files als Interface für 3D Map generierung
    - im Multiplayer, Fog of War Mechanik und highlight für sichtbares Areal der Spieler

    \section{UI und HUD}
    Eigene Ansicht:
    - Eigene HP, Rüstungsklasse und kontrollierte NPCs, sowie Statuseffekte als HUD
    - Menü Aufruf öffnet Charakter sheet und erlaubt so eigene Stats zu verändern
    - Menü für eigene Item (eigenes Inventar)
    - Menü für eigene Spells

    \section{Graphics/Visuals}
    Spieler ansicht:
    - Herz, welches die aktuellen HP anzeigt
    - XP Balken
    - Schild, das den aktuellen Verteidigungswert anzeigt
    - aufzählung der jeweiligen Stärke, Intelligenz, Konstitution, Weisheit, Charisma
    - evtl Minibild von kontrollierten NPCs
    - kurzbild von Statuseffekten

    Detaillierte Ansicht:
    - genaue Charaktere Statistiken; Charaktersheet
    - Informationen über Inventar (Inhalt, Platz, usw.)
    - detailierte Statuseffekte
    - kontrollierte NPCs in Namensliste

    \section{Soundeffects}
    - Link zu Spotify/Youtube um Musik für Atmosphäre zu steuern
    - evtl Soundeffekte für bestimmte Spiel Events, Explosionen, Angriffe

\end{document}
