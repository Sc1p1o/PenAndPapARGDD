Die Entwicklung von PenAndPapAR ist in mehreren Schritten geplant.\ Diese werden hier nochmal niedergeschrieben.\ Eine
Notiz zeigt auch den aktuellen Stand.

\subsection{Planungsphase}\label{subsec:planning}
In der Planungsphase wird das Projekt geplant und organisatorische Aufgaben gestartet.\ Diese Phase hat folgende
Meilensteine und Ziele:

\begin{itemize}
    \item Aufgaben- und Rollenverteilung
    \begin{itemize}
        \item Teamlead: Martin Miller
        \item Technische Leitung: Jannis Kerz
        \item Zweite Technische Leitung: Salem Zin Edin Naser
        \item Design Leitung: Cassandra Monika Lenk
    \end{itemize}
    \item Festlegung der Zielplattform
    \begin{itemize}
        \item Primär Ziel: Holo Lens
        \item Nice To Have: Alle VR/AR fähigen Geräte
    \end{itemize}
    \item Einrichtung von Versionskontrollsystemen (Git)
    \begin{itemize}
        \item Anfangs Unity Version Control, später GitHub
        \item \href{https://github.com/Sc1p1o/PenAndPapARGDD}{Hauptspiel und GDD}
        \item \href{https://github.com/Sc1p1o/PenAndPapARDB}{Implementierung der Datenbank}
    \end{itemize}
    \item Bestimmung der primären Entwicklungstools und Frameworks
    \begin{itemize}
        \item Entwicklung: Unity als Game Engine
        \item Persistierung: DjangoDB für das Backend
        \item Kommunikation zwischen DB, Unity und Charakterquelle: ReST API
        \item Server Backend für Multiplayer: ausstehend
    \end{itemize}
    \item Erstellung eines MockUps
\end{itemize}

\subsection{Einarbeitung}\label{subsec:getting_started}
In der Einarbeitung wird das Team auf die kommenden Herausforderungen vorbereitet und instruiert und der Grundstein
für das Projekt gelegt.\ Dies lässt sich insgesamt auf folgende Meilensteine unterteilen:
\begin{itemize}
    \item Aufsetzen des Gits und der .gitignore für Unity, um große Uploads zu vermeiden.
    \item Einrichtung von Unity für ein HoloLens Projekt
    \item Einrichtung der HoloLens
    \item Verbindung der HoloLens mit Unity
\end{itemize}


\subsection{Das MVP}\label{subsec:mvp}
Das MVP sollte primär Inventar, Stats und Notes haben und mit MOCK-UP daten einen Character darstellen, um eine grobe
Kernfunktion zu zeigen und die Idee grob zu visualisieren
Dazu gehören die folgenden Features:
\begin{itemize}
    \item Charakter Stats
    \begin{itemize}
        \item Layer 1 mit funktionierenden Conditions
        \item Layer 2 mit zusätzlich Attributen und Haupt Charakter Informationen
        \item Layer 3 mit ausführlichen Charakter-Informationen
    \end{itemize}
    \item Das Inventar mit Buttons für einzelne Items
    \begin{itemize}
        \item Mock Up Items
        \item Informationen zu enthaltenen Items abrufbar
    \end{itemize}
    \item Eine Oberfläche um Notizen zu erstellen
    \begin{itemize}
        \item Neue Note Tabs erstellbar
        \item Notizen in Note Tabs einfügbar
        \item Layer 1 als Button
        \item Layer 2 mit Notizen
    \end{itemize}
    \item hard Coded Charakter
    \begin{itemize}
        \item Charakter Eigenschaften werden im Code erstellt
        \item Eigenschaften sind manipulierbar in der UI
    \end{itemize}
\end{itemize}


\subsection{Persistierung und Importierung}\label{subsec:database_and_import}
Die Persistierung der erstellten Charaktere ist ein Kern Element für unsere Funktion, da ein Charakter, welcher nicht
gespeichert werden kann, am Ende nicht die Papiercharakterbögen ersetzt.
Die Datenbank benötigt folgende Funktionen:
\begin{itemize}
    \item Modellierung
    \begin{itemize}
        \item Charakter ist mit Models in Django repräsentiert.
        \item Charakter kann aus der Datenbank eine JSON ausgeben.
        \item Charakter wird durch eine JSON eingelesen und in Datenbank gespeichert.
    \end{itemize}
    \item Integritätssicherung
    \begin{itemize}
        \item Eingelesene JSON wird auf korrektes Format geprüft.
        \item Ausgegebene JSON ist standardisiert.
        \item Charaktere im falschen Format werden nicht gespeichert.
    \end{itemize}
    \item Schnittstelle nach aussen
    \begin{itemize}
        \item GET, POST und PUT funktionieren und sind über ReST abrufbar.
        \item Operationen erfüllen Integritätsbedingungen.
        \item Bilderkennung um ein Sheet auszulesen und das JSON Schema zu füllen
        \item API Schnittstelle, um einen Charakter aus \textit{DnD-Beyond} auszulesen.
    \end{itemize}
\end{itemize}


\subsection{Der Mitspieler}\label{subsec:the_player}
Jeder Spieler hat Zugriff auf die Basis Spieler Tools, diese umfassen folgende Features:
\begin{itemize}
    \item Verbindung zum Backend
    \begin{itemize}
        \item Charakter wird von Datenbank in Unity geladen
        \item Charakter wird aus Unity in Datenbank gespeichert
        \item Datenbank ist stets erreichbar unter standardisierter Addresse
        \item Datenbank ist um Spielerverzeichnis erweitert
    \end{itemize}
    \item HUD ist vollständig
    \begin{itemize}
        \item Stats sind fertig
        \item Inventar ist fertig
        \item Notizen sind fertig
        \item Charakter kann importiert werden aus dem HUD aus
        \item Aktionen und Zauber sind im HUD integriert
    \end{itemize}
    \item HUD Design ist fertig
    \begin{itemize}
        \item Inventar hat grafische Icons und Grafiken
        \item Stats haben Icons und Grafiken+
        \item Notizen haben Icons und Grafiken
        \item Aktionen und Zauber haben Icons und Grafiken
        \item Importierung eines Charakters hat Icons und Grafiken
    \end{itemize}
    \item Charakter vollständig
    \begin{itemize}
        \item Text im HUD ist minimiert auf das nötigste
        \item HUD Icons erklären durch Aussehen, was sie darstellen.
        \item Bilderkennung funktioniert stabil und zuverlässig
    \end{itemize}
\end{itemize}


\subsection{Der Spielleiter}\label{subsec:gamemaster}


\subsection{Die Gefährten}\label{subsec:npcs}


\subsection{Ein Held kommt selten allein}\label{subsec:connecting_friends}


\subsection{Weltenbummler}\label{subsec:its_a_whole_new_world}
