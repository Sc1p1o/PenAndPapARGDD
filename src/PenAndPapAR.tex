%! Author = mmiller
%! Date = 10.12.24

% Preamble
\documentclass[11pt]{article}

% Packages
\usepackage{amsmath}
\usepackage{titling}
\usepackage{graphicx}



% Document
\begin{document}

    \title{Pen and PapAR}
    \author{Martin Miller, Jannis Kerz, Cassandra Lenk und Salem Zin Ider Naser}

    \maketitle
    \newpage

    \tableofcontents
    \newpage

    \section{TODOs}\label{sec:todos}
    Folgende Dinge sind im GDD bereits dokumentiert, aber noch nicht im Spiel implementiert:

    \begin{itemize}
        \item Icons und Graphics einbauen (Merge ausstehend, siehe Git)
        \item einlesen von Texten via HoloLens (begonnen)
        \item generierte 3D Karte (ausgearbeitet, wartend auf Abhängigkeiten)
        \item Multiplayer
        \item Spielleiter Ansicht
        \item Aktionen und Zauber Menü
    \end{itemize}


    \section{Idee}\label{sec:chapter_idea}
    Die Idee hinter dem Spiel \textit{Pen\&PapAR} ist es, den seit Serien wie
\textit{Stranger Things} oder \textit{The Big Bang Theorie} immer beliebteren und
spätestens mit dem Hype um das Spiel \textit{Baldur's Gate 3} und dem Film
\textit{Dungeons und Dragons - Ehre unter Dieben} auch im Mainstream angekommenen
Pen\&Paper Klassiker \textit{Dungeons and Dragons} so wie weitere Pen\&Paper Systeme nun
auch in die heutige Zeit zu holen.\newblock

Dabei ist es uns wichtig das Spiel nicht komplett in die digitale Welt zu holen, sondern
den besten Part, als soziales Zusammenkommen möglichst beizubehalten.
Genauso wichtig ist es den Spielern nichts Digitales aufzuzwingen, dafür müssen
die Spieler die gleichen Freiheiten haben, wie bei einem klassischen Spiel.
Das bedeutet, dass das Charakter-Sheet des Spielers sowohl digital 100\% editierbar ist,
als auch, dass Spieler ihren Grad an digitalisierung frei entscheiden können müssen und
alle potenziell analog ausgeführten Aktionen von Spielern selbst digital nachgetragen werden
können müssen, falls nötig bzw.\ gewünscht.\newblock

Der Vorteil einer digitalisierten Version ist aber vor allem, der Platz, dies sollte auch
Hauptaugenmerk unseres Spiels sein.
Charakter-Sheets sind oft viel zu klein um seinen ganzen Loot und all die mächtigen Angriffe,
Zauber und Fähigkeiten niederzuschreiben, die man im Laufe seiner Kampagne gesammelt hatte.
Dazu sind diese meist mehr als eine Seite lang und sobald man ein Haustier oder einen Begleiter
hat, kann das schnell in einem großen Papier-Chaos enden, grade in der Rolle als Spielleiter
hat man schnell viele Bücher, Regelwerke, und NPC-Sheets auf dem Tisch liegen.\newblock

Mit unserem Spiel ist es einfach und übersichtlich all diese Dinge im Blick zu behalten und
dabei den Tisch freizuhalten.
Mit einem Klick kann man eine Liste aller kontrollierter NPCs
ansehen und deren Details nachschauen oder Informationen zu einem bestimmten Regelwerk oder Notizen
nachzuschauen.
Somit hat man viel mehr Platz am Tisch für die Snacks, Würfelwürfe und die sorgfältig bemalten
Minis.\newblock

Es gibt oft zwar bereits digitale Lösungen, diese beschränken sich aber oft auf Laptops
oder sind nur unter einem kostenpflichtigen Abo abrufbar.
Außerdem ist es größtenteils schwer bei den offiziellen Anbietern fremde sogenannte \textit{Homebrew}-Mechaniken
und Items hinzuzufügen.
Dies soll bei unserem Spiel soweit möglich vereinfacht werden, sodass unsere Spieler nicht erst kompliziert
ein Item konfigurieren müssen und es einfügen müssen, damit es verwendbar wird.
Spieler sollen frei entscheiden können, ob sie die Fähigkeiten eines solchen Gegenstands einfach
in der Beschreibung erläutern oder sich die Mühe machen wollen, die Mechanik vollständig in das
Spiel zu integrieren.
Beide wege sollen dabei gleichermaßen intuitiv sein und die Funktion der Waffe ähnlich
übersichtlich darstellen.\newblock

Wenn all diese Aspekte fertiggestellt sind, ist das nächste Ziel nach dem Komfort und Platzsparen, die Immersion des
Spiels zu erhöhen.\newline
Der erste Part hierbei wird es sein, eine 3D-Karte zu erstellen, welche das aktuelle Schlachtfeld, den Dungeon oder
das aktuelle Areal in 3D darstellt und den Spielleiter interaktiv Punkte auswählen lässt, um Spieler so auf eine
Regionskarte der eine genauere Karte öffnet, auf der die Spieler danach ihre Minis bewegen können.
Dies setzt natürlich für alles eine solche Karte in digitaler Form voraus.
Sollte der Spielleiter keine solche Karte eingestellt haben, kann er dennoch auch über das Menü sofort
eine Kampfkarte seiner Wahl öffnen.
Dieses Feature soll allerdings, für den Fall, dass die Spieler eine selbst gemalte Karte bevorzugen, auch
ausschaltbar sein.
Oberste Priorität ist der Komfort der Spieler ihr Spiel nur so weit zu digitalisieren, wie sie es sich wünschen.
Unser Spiel soll sie dabei so wenig wie möglich zwingen, Elemente des Pen\&Papers zu digitalisieren.\n

    \section{Gameplay}\label{sec:chapter_gameplay}
    \subsection{Core Game Loop}\label{subsec:chapter_gameloop}

Beim Start lädt das Spiel automatisch den Character aus der Datenbank und füllt die verschiedenen Charaktereigenschaften
mit den entsprechenden Informationen aus.\ Dann können die Spieler auch bereits loslegen und sind spielbereit.\ Durch
Antippen der einzelnen Menüs können sie die kleinen Icons erweitern, um ausführlichere Charaktereigenschaften zu erhalten
und abhängig davon in welchem Layer sie sich befinden, können sie dann auch mehr Eigenschaften bearbeiten.
Auch können sie ihr Inventar verwalten und sich eigene Notizen machen, indem sie auf die jeweils anderen dazugehörigen
Menüs tippen.\ Wenn die Spieler ihre Menüs erstmal aus dem Weg haben wollen, können sie diese durch gedrückt halten auf
das Icon im ersten Layer in 3D Objekte verwandeln und dann im Raum abstellen, der ihnen besser passt.

Auch können sie mit einer eigenen API ihren \textit{DnD-Beyond} Charakter verwenden.\ Da \textit{DnD-Beyond} nach unserem
Wissensstand aktuell allerdings keine eigene öffentlich zugängliche API zur Verfügung stellt, muss der Charakter auf
\textit{DnD-Beyond} öffentlich einsehbar sein.\ Die Datenbank erlaubt dabei auch das Speichern mehrerer Charaktere, da
diese automatisch mit einer ID versehen werden.\ Der Index reicht für 9.999 Charaktere pro Spieler.

Aktionen, Zauber und Feats müssen aktuell allerdings noch in das Feld eingefügt werden oder unter Notizen notiert werden,
da aktuell keine Oberfläche dafür implementiert ist.\ Dies sollte noch implementiert werden, so wie auch die 3D Map
und der die Spielleiter Ansicht.

Da diese Dinge allerdings erstmal auch eine Multiplayer Verbindung bedingen, um ihre Sinnhaftigkeit zu erfüllen wurden
diese Dinge noch nicht implementiert, allerdings ist ihre Konzeption zumindest in Teilen bereits vollzogen.

\subsection{Steuerung}\label{subsec:gameplay_controls}

\subsubsection{Allgemeine Steuerung}
Im Idealfall ist das Spiel vollständig mit Handbewegung steuerbar und bedarf lediglich die HoloLens.
Das gedrückt halten eines Menüs verwandelt es in ein 3D Item, das durch den Raum gezogen werden und abgelegt werden kann
 und dann beim Antippen alle informationen anzeigt.
Erneutes Doppeltippen des 3D Objekts soll dieses dann wieder verschwinden lassen und stattdessen wieder das vorherige
HUD-Menü erscheinen lassen.
Die Booleschen und Zahlen werte sollen mit der Hand manipuliert werden.\ Die Manipulation von Textwerten soll durch
Laden aus der \textit{DnD-Beyond} API oder durch Einlesen des Textes über die Kamera der HoloLens passieren.

\subsubsection{Kartensteuerung}
Die generierte Karte soll durch auseinander ziehen und zusammen ziehen von Daumen und Zeigefinger vergrößert oder
verkleinert werden.
Durch langes Antippen kann die Map dann verschoben werden und mit der zweiten Hand waagerecht gedreht werden,
wenn die Karte deaktiviert werden soll, kann der Spielleiter die Karte gedrückt halten und nach oben wischen.
Durch Antippen und links und rechts wischen wird die nächste Map in der Rotation geladen und generiert.
Durch Antippen eines Quadrats wird es dann möglich sein, auch seine eigenen Figuren zu platzieren und über die Karte zu
bewegen.

\subsubsection{PC Steuerung}
Für das einfachere Debuggen und weil auch nicht jeder eine HoloLens zur Verfügung hat, sollte das Spiel so konzipiert sein,
das das Menü und die Map auch mit Maus und Tastatur anzeigbar ist.\ Dabei simuliert der Mausklick ein Antippen mit dem Finger.
Das Mausrad wird dann für eventuelle Drehbewegungen und Zooms verwendet.

\subsection{Der Spieler}\label{subsec:gameplay_player}
Die Spieler können ihre Charaktere entweder aus \textit{DnD-Beyond} importieren, ihren auf Papier geschriebenen Charakter
einscannen oder die werte komplett selbst anpassen und daraus einen Charakter erstellen.
Der Spieler kann mit anderen Spielern in eine Session gehen und so eine Party starten.
Dabei soll dann einer der Spieler den Spielleiter übernehmen und ein erweitertes HUD haben, mit dem er weitere Optionen
hat für die Session um ein vorher importiertes Kartenset zu laden und NPCs auf diesen zu platzieren.
Innerhalb dieser Session hat der Spielleiter als eine Art Admin auch eingeschränkte Rechte die Charaktere der anderen
Spieler zu beeinflussen durch das Zuweisen von Status Effekten zum Beispiel.

\subsubsection{Der Mitspieler}
Jeder Spieler hat Zugriff auf das Standard HUD mit der Ansicht über einen Character.\ Auch der Spielleiter hat diese
Ansicht, weil es grade bei kleinen Gruppen auch dazu führen kann, das der Spielleiter auch gleichzeitig einen Charakter
mitspielt.
Die Ansicht enthält Informationen über den Charakter, das Inventar, Notizen und ein Menü für Aktionen und Zauber des
Characters.
\newblock

    \section{Systems}\label{sec:systems}
    - Hololens für AR
    - Server für lokalen Mehrspielermodus
    - JSON Files als Interface für 3D Map generierung
    - im Multiplayer, Fog of War Mechanik und highlight für sichtbares Areal der Spieler

    \section{UI und HUD}\label{sec:ui-und-hud}
    Eigene Ansicht:
    - Eigene HP, Rüstungsklasse und kontrollierte NPCs, sowie Statuseffekte als HUD
    - Menü Aufruf öffnet Charakter sheet und erlaubt so eigene Stats zu verändern
    - Menü für eigene Item (eigenes Inventar)
    - Menü für eigene Spells

    \section{Graphics/Visuals}\label{sec:graphics/visuals}
    Spieler ansicht:
    - Herz, welches die aktuellen HP anzeigt
    - XP Balken
    - Schild, das den aktuellen Verteidigungswert anzeigt
    - aufzählung der jeweiligen Stärke, Intelligenz, Konstitution, Weisheit, Charisma
    - evtl Minibild von kontrollierten NPCs
    - kurzbild von Statuseffekten

    Detaillierte Ansicht:
    - genaue Charaktere Statistiken; Charaktersheet
    - Informationen über Inventar (Inhalt, Platz, usw.)
    - detaillierte Statuseffekte
    - kontrollierte NPCs in Namensliste

    \section{Soundeffects}\label{sec:soundeffects}
    - Link zu Spotify/Youtube um Musik für Atmosphäre zu steuern
    - evtl Soundeffekte für bestimmte Spiel Events, Explosionen, Angriffe

\end{document}
