%! Author = Cassandra Monika Lenk
%! Date = 14.03.2025

% Preamble
\documentclass[11pt]{article}

% Packages
\usepackage{amsmath}
\usepackage{textcomp}

% Document
\begin{document}
    \title{Pen and PapAR \\ \large \textminus Cassandra Lenk Projektbericht \textminus}

    \maketitle

    \section{Projektarbeit}\label{sec:chapter_projectwork_clenk}

    Das Modul „Einführung in die Virtuelle und Erweiterte Realität“ hat einen tiefen Bezug sowohl zur Videospiel-Programmierung als auch zur Arbeit mit VR/AR-Brillen sowie den Arbeitsumfeldern Unity und Unreal Engine aufgezeigt. Wir erhielten innerhalb der Vorlesungen wertvolle Informationen darüber, wie unser Spiel besonders nutzerfreundlich und „spielbar“ gestaltet werden kann. Ein zentraler Punkt war die Auseinandersetzung mit der Wahrnehmung von Farben sowie der Einfluss von Kalibrierung und den interaktiven Elementen eines Spiels auf die Spielerfahrung und den menschlichen Körper. Besonders die Frage, wie visuelle Elemente in VR/AR-Umgebungen wahrgenommen werden, stellte sich als äußerst relevant heraus, da hier die Gestaltung eine direkte Auswirkung auf das Wohlbefinden und die Immersion der Spieler hat.
    Ein weiterer wichtiger Aspekt war die Einführung in Unity und das Arbeiten mit den damit verbundenen Tools. Dies war für mich eine neue Erfahrung, da ich bisher vorwiegend in traditionellen Programmiersprachen gearbeitet habe, während die meisten visuellen Umsetzungen, wie z. B. mit HTML, CSS oder ähnlichem, durchgeführt wurden. Die Einführung in Unity und speziell in die Entwicklung für VR/AR-Plattformen hat mir neue Perspektiven eröffnet und mir einen praktischen Einstieg in die Entwicklung interaktiver Anwendungen ermöglicht.
    Das Modul „Einführung in die Virtuelle und Erweiterte Realität“ hat mir somit einen wertvollen Einblick in ein völlig neues Feld der Spieleentwicklung und interaktiven Medien verschafft. Zu Beginn war die Aufgabe, ein solches AR-Projekt zu planen. Die Planungserfahrungen, die wir aus vorherigen oder parallelen Modulen wie dem Softwareprojekt hatten, konnten wir hier zwar nutzen, jedoch stellte sich die Planung für dieses Modul als deutlich anders heraus. Besonders die Erstellung eines Game Design Dokuments (GDD) war für mich eine völlig neue Herausforderung, da ich noch nie zuvor ein solches Dokument verfasst hatte. In diesem Dokument wurden nicht nur die Spielmechaniken und die technische Umsetzung beschrieben, sondern auch der nutzerzentrierte Ansatz für die Gestaltung des Spiels sowie die Benutzererfahrung (UX).
    Zu Beginn des Projekts lag die Erstellung und Planung hauptsächlich auf den Schultern von zwei Personen, da die anderen Teammitglieder erst später zu dem Projekt stießen. Dies stellte uns vor die Herausforderung, große Verantwortung zu übernehmen und Entscheidungen für das gesamte Team zu treffen, aber auch die Chance, das Projekt von Grund auf zu gestalten. In dieser Phase übernahm ich erstmals vorwiegend die Design-Rolle, was für mich eine spannende Abwechslung darstellte. Normalerweise war ich eher in der reinen Umsetzung von Funktionalitäten tätig, doch in diesem Fall wollte ich die Gelegenheit nutzen, mich intensiver mit der Optik und dem visuellen Design eines Produkts auseinanderzusetzen.
    Im Laufe des Projekts wurde mir jedoch auch bewusst, wie anspruchsvoll es ist, ein Projektdesign zu entwerfen, insbesondere ohne vorherige Erfahrungen in diesem Bereich. Es galt, nicht nur eine funktionale Benutzeroberfläche zu schaffen, sondern auch die Ästhetik so zu gestalten, dass sie zur VR/AR-Erfahrung passt und gleichzeitig intuitiv für den Nutzer bleibt. Wir haben regelmäßig im Team meine Designvorschläge besprochen und diese gemeinsam angepasst, bis wir mit dem finalen Design zufrieden waren.
    Ein entscheidender Punkt war es, ein Design zu entwickeln, das für unser Projekt besonders geeignet ist, da es sich hierbei nicht um ein „reines Spiel“ handelt, sondern eher um eine Spielunterstützung, die das Spielen durch zusätzliche Informationen und Interaktionen erleichtern soll. Diese Unterscheidung brachte zusätzliche Komplexität mit sich, da das Design funktional und zugleich praktikabel für den Nutzer sein musste. Ein weiterer wichtiger Aspekt war die Zugänglichkeit der Informationen in der erweiterten Realität, sodass diese nicht nur informativ, sondern auch schnell und einfach zugänglich sind.
    Zur Umsetzung selbst haben wir, wie auch aus den Berichten meiner Teamkollegen ersichtlich, jeweils an verschiedenen Interfaces gearbeitet. Mein Beitrag war hierbei insbesondere die Umsetzung des Inventars, das für die HoloLens 2 voll-interaktiv gestaltet werden sollte. Dabei stellte sich heraus, dass die Entwicklung eines solchen Systems deutlich komplexer war als erwartet. Wir hatten geplant, dass die Inhalte des Inventars eigene Pop-up-Fenster haben, die durch einen Knopfdruck geöffnet werden können, was uns vor technische und visuelle Herausforderungen stellte. Ohne umfassende Unity-Kenntnisse empfand ich die Aufgabe, das Inventarsystem funktional und gleichzeitig benutzerfreundlich zu gestalten, als besonders anspruchsvoll.
    Zunächst funktionierten die Buttons nicht korrekt, und hinzugefügte Inhalte wurden falsch angezeigt, was zu einigen Frustrationen führte. Dennoch war es mir möglich, die grundlegenden Funktionen der Pop-up-Fenster für die Items zu implementieren, sodass zu jedem existierenden Item ein Name, eine Beschreibung und ein Gewicht angezeigt werden können. Leider war es mir zu diesem Zeitpunkt nicht mehr möglich, die Funktion zu realisieren, bei der der Spieler direkt im Spielmodus ein neues Item inkl. aller Randdaten hinzufügen kann. Diese Funktionalität sollte später jedoch nachgeholt werden.
    Abgesehen von der funktionalen Umsetzung habe ich mich auch um die Icons für die Conditions gekümmert. Ich habe passende Icons herausgesucht, sie in Splines umgewandelt und ins Projekt integriert. Dies stellte sich als relativ einfache, aber wichtige Aufgabe heraus, die dem Projekt mehr Optik und eine visuelle Struktur verlieh.

    Zur Teamarbeit lässt sich festhalten, dass unsere Meetings insgesamt sehr produktiv und effektiv waren, was wesentlich dazu beigetragen hat, dass das Projekt kontinuierlich vorangekommen ist. Die Zusammenarbeit im Team war stets von einem hohen Maß an Engagement und Aktivität geprägt, was ein sehr angenehmes und kooperatives Arbeitsumfeld geschaffen hat. Jeder von uns brachte seine Ideen und Vorschläge ein, was zu einer ständigen Weiterentwicklung des Projekts führte. Besonders hervorzuheben ist, dass wir die Führung unserer Protokolle sehr gut umgesetzt haben, wodurch wir jederzeit eine klare Übersicht über die bereits besprochenen Punkte und die nächsten Schritte hatten.
    Ein großer Vorteil war, dass wir regelmäßig die Möglichkeit hatten, uns in Person zu treffen. Diese Treffen nutzten wir nicht nur dazu, aktuelle Fortschritte zu präsentieren, sondern auch um gemeinsam Lösungen für Probleme zu erarbeiten und neue Ideen zu diskutieren. Besonders effektiv war es, dass jeder seine bisherigen Ausarbeitungen in Form von kurzen Präsentationen vorstellte, was den Austausch förderte und sicherstellte, dass alle Teammitglieder auf dem gleichen Stand waren. Wir konnten so nicht nur technische Details besprechen, sondern auch das Design und die Benutzererfahrung gemeinsam weiterentwickeln.
    Ein weiterer wichtiger Punkt, der unsere Arbeit bereichert hat, war der Einsatz von großen interaktiven Tafeln während unserer Treffen. Diese ermöglichten es mir, meine Designvorschläge und Konzepte direkt zu projizieren und visuell zu präsentieren. In Kombination mit meinem Laptop oder Tablet konnten wir so gemeinsam das Design verfeinern und Änderungen in Echtzeit vornehmen. Diese Art der Zusammenarbeit war besonders effektiv, da sie uns ermöglichte, Designentscheidungen schnell zu treffen und Missverständnisse zu vermeiden.
    Leider mussten wir gegen Ende des Projekts feststellen, dass die Arbeitsmoral in Richtung der Prüfungsphase hin deutlich nachgelassen hat. Obwohl wir zu diesem Zeitpunkt einen wichtigen Punkt im Projektablauf erreicht hatten, fielen die Teammeetings zunehmend kürzer aus und wurden häufiger auf online verlegt, was die Kommunikation und den Austausch erschwerte. Ein wesentlicher Grund dafür war, dass andere Prüfungen und Projekte aus dem 5. Semester zunehmend Priorität hatten, wodurch die AR-Anwendung in den Hintergrund geriet. Diese Verschiebung hatte zur Folge, dass Deadlines nicht immer eingehalten wurden und die Qualität mancher Arbeiten nicht dem ursprünglich gesetzten Standard entsprach.
    In dieser Phase kam es auch dazu, dass bestimmte Aufgaben vergessen oder nicht vollständig umgesetzt wurden, was vor allem die Funktionalität der HoloLens-Anwendung beeinträchtigte. Da wir nicht immer in der Lage waren, die notwendigen Komponenten rechtzeitig zu integrieren, konnten verschiedene Features nicht getestet werden, was wiederum die Fertigstellung des Projekts verzögerte. Zudem hatte der Mangel an regelmäßigen und produktiven Treffen zur Folge, dass wichtige Diskussionen und Entscheidungen nicht immer im Detail getroffen werden konnten, was den Fortschritt des Projekts hemmte.
    Ein weiteres Problem trat auf, als es darum ging, unsere individuellen Branches auf den Master-Branch zu mergen. Besonders problematisch waren die Konflikte in den Szenen, die beim Zusammenführen der verschiedenen Branches auftraten. Diese Konflikte mussten teilweise manuell gelöst werden, was zusätzlichen Zeitaufwand verursachte und den Prozess des Mergens erheblich verlangsamte. Anstatt dass das Mergen eine schnelle und effiziente Aktion war, wurde es durch diese Konflikte zu einer zeitintensiven Aufgabe, die uns wertvolle Zeit raubte, die wir besser in die eigentliche Entwicklung hätten investieren können.
    Durch diese technischen Probleme und Konflikte mit Unity sind uns leider eine beträchtliche Menge an Entwicklungszeit entfallen, da wir uns verstärkt auf das Lösen dieser Herausforderungen konzentrieren mussten. In der Rückschau hätte es vermutlich geholfen, wenn wir mehr Zeit für die Vorbereitung und Planung von Unity-Projekten aufgewendet hätten, um solche Konflikte von vornherein zu minimieren. Zudem hat uns die nicht optimale Kommunikation in den letzten Wochen des Projekts in gewisser Weise ausgebremst, da wichtige Informationen und Updates teils nicht rechtzeitig ausgetauscht wurden.
    Dennoch haben wir es geschafft, trotz dieser Rückschläge ein Produkt zu entwickeln, das die Grundfunktionen des AR-Projekts umsetzt und uns wertvolle Erfahrungen in der Zusammenarbeit und der technischen Umsetzung verschafft hat. In Zukunft könnten wir durch eine intensivere Zeitplanung und eine frühzeitigere Einbindung aller Teammitglieder in den Entwicklungsprozess sicherlich eine noch effizientere und fehlerfreiere Projektumsetzung erreichen.
    Zusammenfassend lässt sich sagen, dass dieses Modul für mich eine wertvolle Erfahrung war, insbesondere hinsichtlich der praktischen Anwendung von Unity, der Gestaltung für VR/AR und der Entwicklung von Interfaces, die nicht nur funktional, sondern auch intuitiv und optisch ansprechend sind. Trotz der Herausforderungen konnte ich neue Fähigkeiten erlernen und meinen Horizont in der Spieleentwicklung und interaktiven Mediengestaltung erweitern.


\end{document}

