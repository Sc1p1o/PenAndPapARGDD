%! Author = Cassandra Monika Lenk
%! Date = 14.03.2025

% Preamble
\documentclass[11pt]{article}

% Packages
\usepackage{amsmath}
\usepackage{textcomp}

% Document
\begin{document}
    \title{Pen and PapAR \\ \large \textminus Cassandra Lenk Projektbericht \textminus}

    \maketitle

    \section{Projektarbeit}\label{sec:chapter_projectwork_clenk}

    Das Modul Einführung in die Virtuelle und Erweiterte Realität hat einen sowohl einen Bezug
    zum Videospiel-Programmierung als auch den Bezug zur Arbeit mit VR/AR-Brillen,
    sowie den Arbeitsumfeldern Unity/Unreal Engine gezeigt.

    Wir haben innerhalb der Vorlesungen Informationen erhalten, wie unser Spiel besonders
    Nutzerfreundlich und „Spielbar“ wird. Beispielweise unter Beachtung von Farbwahrnehmung
    sowie, wie das Spiel und die Kalibrierung Einfluss auf die Spielerfahrung und den z
    menschlichen Körper hat.

    Die Einführung in Unity und das Arbeiten mit den damit verbundenen Tools war ebenfalls eine
    neuere Erfahrung, da ich bisher vorwiegend rein programmiert habe und die meisten visuellen
    Umsetzungen mit beispielweise HTML, CSS u.Ä. stattgefunden haben.

    Einführung in die Virtuelle und Erweiterte Realität hat hiermit einen Einblick in ein neues Feld geboten.
    Zunächst war unsere Aufgabe ein solchen AR-Projekt erstmals zu Planen. Planungserfahrungen
    haben wir bereits aus vorherigen beziehungsweise parallelen Modulen,
    wie dem Softwareprojekt. Dennoch war die Planung für dieses Modul deutlich anders,
    da ich noch nie vorher ein Game Design Dokument erstellt habe.
    Erstellung und Planung fiel am Anfang ebenfalls nur auf zwei Personen, da unsere anderen
    Teammitglieder erst später Teil des Projekts geworden sind.
    Erstmalig für mich selbst, habe ich in diesem Projekt vorwiegend die Designrolle übernommen.
    Ich wollte die Möglichkeit nutzen aus meiner gewohnten Rolle des reinen Umsetzens von Funktionalitäten
    herauszutreten und intensiver an der Optik eines Produkts arbeiten.
    An dieser Stelle ist mir auch aufgefallen, wie anspruchsvoll es ist ein solches Projekt-Design
    zu entwerfen, vor allem ohne vorherige Kenntnisse, wie es funktioniert.
    Wir haben regelmäßig im Team meine Design Vorschläge besprochen und gemeinsam überarbeitet, bis wir
    zufrieden mit dem finalen Design waren.

    An diesem Punkt ist auch festzuhalten, dass es schwierig war, eine Idee zu finden, wie man ein Design
    für unser Projekt speziell umsetzten kann, da es sich hierbei ja nicht um ein reines Spiel, sondern eher
    um eine Spielunterstützung handelt, die das Spielen erleichtern soll. In diesem Zusammenhang sollten
    unsere gewünschten Interfaces auch praktikabel für den Nutzer selbst sein.
    Zur Umsetzung selbst, wie wahrscheinlich auch aus den Texten meiner Teammitglieder zu lesen,
    haben wir jeweils an Interfaces gearbeitet.

    Ich habe an der Umsetzung unseres Inventars gearbeitet und dabei festgestellt,
    wie komplex es ist ein Inventar für die HoloLens 2 zu erstellen, welches voll-interaktiv ist,
    da wir geplant haben, dass Inhalte des Inventars ihre eigenen Fenster haben, welche via Knopfdruck
    geöffnet werden können.
    Die Funktionalität und Erweiterbarkeit fand ich ohne große Unity-Kenntnisse unglaublich anspruchsvoll und
    ich hatte Probleme mit der Visualisierung der Umsetzung.
    Zunächst genutzte Knöpfe haben nicht korrekt funktioniert, oder hinzugefügte Inhalte wurden falsch
    angezeigt, was für Frustration sorgte.
    Ebenfalls habe ich an den Icons für die Conditions gearbeitet und diese herausgesucht und entsprechend
    als Splines zum Projekt hinzugefügt, was sich als relativ simpel herausgestellt hat und dem Projekt damit
    auch etwas mehr Optik gegeben hat.

\end{document}

