%! Author = Martin Miller
%! Date = 14.03.2025

% Preamble
\documentclass[11pt]{article}

% Packages
\usepackage{amsmath}
\usepackage{textcomp}

% Document
\begin{document}
    \title{Pen and PapAR \\ \large \textminus Martin Miller Projektbericht \textminus}

    \maketitle

    \section{Aller Anfang ist schwer...}\label{sec:chapter_beginning}
    Das Projekt selbst war meine eigene Vision, allerdings war der Start des Projekts aus mehreren Gründen doch sehr
    holprig.\ Der erste Plan war es mit der Unreal Engine 5 zu arbeiten, da ich mit dieser insgesamt mehr Erfahrung hatte.
    Nach mehreren Ansätzen und mehreren Versuchen, ein HoloLens Projekt aufzusetzen, stiessen wir auf mehrere Probleme.
    Zum einen stiessen wir, damals nur Cassandra Lenk und ich, auf einen Artikel, in welchem Unreal den Support für die
    HoloLens 2 beendet hatte.
    Zum Anderen hatten wir Versuch eine andere ältere Version zu verwenden das Problem, das diese nicht auf den Computern
    in der HTWK verfügbar war.

    Somit mussten wir nochmal neu starten und haben das Projekt kurz vor Weihnachten auf Unity begonnen.\ Zu dieser Zeit
    kamen zwei weitere Personen ins Team.\ Salem Zin Ider Nasser und Jannis Kerz stiessen zu uns und mit Jannis hatten
    wir jemanden an Bord, der sich mit Unity auskannte.\ Er erstellte ein kleines Tutorial, um einen Einstieg in Unity zu
    erleichtern.\ Darauf hin begann die Arbeit in Unity an dem Projekt PenAndPapAR mit reichlich Verzögerung zum neuen
    Jahr.
    
    \section{Die Arbeit am Projekt}\label{sec:chapter_working_on_penandpapar}
    Mit der neuen Teamgröße war nun auch ein größerer Scope möglich, allerdings war nun auch eine genauere Teamstruktur
    vonnöten.\ Als der Urheber der Idee und einer der erfahrensten DnD Spieler im Team wurde mir die Rolle als Teamlead
    zugeteilt.\ Bei der Einteilung für die ersten Elemente wurde mir und Jannis das Implementieren der generellen
    Statistiken zugeteilt.\ Dabei habe ich Layer 1 und 3 übernommen und Jannis hat Layer 2 übernommen. \ Ich habe
    anschliessend das grobe Design der aller 3 Layer übernommen und mir wurde die Aufgabe zuteil, das GDD und die
    Dokumentation des Projekts zu schreiben.

    Dazu habe ich Anfangs im wöchentlichen Meeting Protokoll geführt, bis etwa Februar, wo ich damit aufhörte, weil
    diese Protokolle sowieso nie gelesen wurden und ich mich mehr auf das Nacharbeiten der noch fehlenden Features in
    unserem Projekt fokussieren wollte.
    Auch habe ich am Anfang des Projekts das Projekt eingerichtet und den Editor für das Implementieren einer HoloLens 2
    Anwendung eingerichtet, so wie das Aufsetzen der Szene und des Gits.\ Letzteres geschah in Zusammenarbeit mit Jannis
    Kerz, nachdem das von mir zuvor aufgesetzte Unity Version Control System sich als unbrauchbar für unser Projekt und
    unseren Workflow herausstellte.
    Dies erforderte auch erstmal das Herausfiltern von dem riesigen Overhead den der Unity Editor 6 mit in die
    Projektdateien brachte.

    Nachdem die allgemeinen \textit{Stats} fertig implementiert waren, habe ich mich darum gekümmert den Charakter in
    einer Datenbank speichern zu können und wieder abrufen zu können und vorerst einen MOCK-Up Charakter in der Schnittstelle
    erstellen lassen, um die Funktionalität der Stats überprüfen zu kommen.
    Allerdings wurde schnell klar, dass das Interface, wie es bei uns aktuell noch erstellt wurde, nicht anklickbar war.
    Es fehlten Button Objekte, wodurch ich mich dann an die Implementierung der Clickables kümmerte und meinen Plan der
    Datenbank anbindung erstmal zur Seite schob.
    Danach begann ich die Einrichtung der HoloLens und musste feststellen, dass das Projekt aus irgendeinem Grund nicht
    sichtbar war in der HoloLens.\ Ich richtete sie ein, aber übergab die Aufgabe, diesen Bug zu beheben dann an Salem Zin
    Ider Nasser und widmete mich stattdessen einem weiteren Problem, das sich abzeichnete.

    Denn es wurde deutlich, dass das mergen in Unity komplexer war als erwartet mit Git.\ So geschah es zum Beispiel bei
    unserem ersten Merge Versuch, das plötzlich die Szenen unbrauchbar waren und Teilweise sogar das Projekt selbst im
    Unity Editor nicht mehr geöffnet werden konnte.\ Somit übernahm ich auch das Beheben etlicher Merge Konflikte und das
    Einführen eines Workflows zum mergen, von zwei Branches aus Unity um zukünftig zu verhindern, das ein Merge die
    Settings und Szenen, die bereits implementiert waren, erneut zerstört.
    Dies geschah dann mit einer Django DB, weil ich mit Django bereits erste berührungspunkte hatte und so einen
    einfacheren Einstieg hatte, um diese Datenbank anschliessend über eine \textit{REST-Schnittstelle} mit dem Unity-Projekt
    zu Verbinden.

    Danach erweiterte ich noch die Sichtbarkeiten der Stats und erweiterte sie um die Funktion auch das dazugehörige 3D-Objekt
    aufzurufen.

    Dies nahm einen großen Teil meiner Zeit in den letzten Februar Wochen ein zusammen mit dem Anpassen der Schnittstelle
    in Unity.\ Nachdem beides fertig war, ging es für mich ab März dann zum Thema implementation der REST Schnittstelle
    selbst in Kommunikation mit Jannis Kerz, damit dieser seine \textit{DnD-Beyond} API an meine REST-API anschliessen
    konnte.

    Zwischendurch hatte ich auch bereits vorarbeit geleistet und mich schlau gemacht über eine mögliche 3D-Map Generierung,
    basierend auf gängigen Dungeon Builder Anwendungen.\ Diese boten nämlich das Exportieren ihrer Maps an, um sie in
    Virtuellen Tabletop Spielen zu nutzen, wie zum Beispiel Forge oder Roll20 es anbietet.
    Die Idee war simpel: Ein 5 foot Quadrat auf der Karte wird dargestellt durch einen 3D-Würfel, welcher auf der
    Oberseite als Textur einen Teil der Map enthielt.
    Die Informationen wären in einer JSON enthalten und so würde man dann eine Map zusammen basteln können.
    Leider fehlten zum Stand jetzt zwei Dinge, um diese Funktion wirklich zu implementieren.
    Die ausgegebenen JSON Daten boten zwar Informationen für \textit{Fog of War} und die angabe, wie gross ein 5 Feet
    Kästchen in Pixeln sein soll auf dem ausgegebenen Image.
    Allerdings waren die 3D-Informationen entweder ganz verloren, oder für uns zwar auffindbar, aber nicht lesbar.

    \section{Was lief gut}\label{sec:chapter_goodthings}
    Gut hatte bei unserem Team die Kommunikation funktioniert und allgemein auch die Hilfsbereitschaft unseres Teams,
    anderen Teammitgliedern.\ Wenn Hilfe benötigt wurde, gab es stets jemanden, der bereit war auszuhelfen.
    Auch die Meetings waren äußerst produktiv und hielten sich in ihrer Länge in Grenzen.
    Team Meetings waren allgemein im Vergleich zu bisherigen Erfahrung sehr Produktiv und brachten Ergebnisse und eine
    klare Arbeitsteilung.\ Am Ende hatte jeder klare Aufgaben zu erledigen.


    \section{Was könnte man verbessern?}\label{sec:chapter_improvables}
    Allerdings gab es durchaus viele Dinge, die in unserem Projekt nicht optimal liefen.\ Angefangen bei den Problemen,
    das Projekt auf Unreal Engine 5 laufen zu lassen zum Beispiel.
    Auch gab es enorme Verzögerungen im Zeitplan anfangs durch Krankheiten und später durch einfach mangelnde
    Kommunikation.\ So hatte ich teilweise Phasen erlebt, wo ich selbst auf Nachfrage keine Informationen zum
    Zwischenstand erhielt und Fristen zugesichert wurden, dann aber mehrfach nicht eingehalten wurden.
    Auch hab es Vorkommnisse, bei denen Teammitglieder ihren Aufgaben nicht nachgingen und diese Aufgaben dann von mir
    oder Anderen Mitgliedern nachgearbeitet werden mussten.\ So waren Merge-Requests Ursprünglich Aufgabe der 2.
    Technischen Leitung und es galt die Prämisse, Merge Konflikte selbstständig zu beseitigen, beides fiel dann
    allerdings öfter auf mich zurück.
    Auch sieht man im GitHub Repo unter Statistiken ein deutliches Ungleichgewicht in der Arbeitsverteilung.
    Zugegebenermaßen hat das Übernehmen der Merge Requests von mir den Aufwand zu meinen Gunsten verzerrt.\ Aber gerade
    die Commits geben einen guten Eindruck der Arbeitsteilung, bzw auch die differenz der Hinzufügenden und Entfernenden
    Contributions (~60.000 bei mir).\ Eine Zahl die sich von anderen Teilnehmern durchaus abhebt.\ Die mir anfangs
    zugesicherte Hilfe bei der Doku oder dem GDD blieb leider komplett aus, was mir nur mehr Zeit weg nahm die ich auch
    für Code nutzen könnte.

    \section{Was habe ich gelernt?}\label{sec:chapter_learnings}
    Mit meiner Arbeit in diesem Projekt habe ich gute Einblicke in das Entwickeln mit Unity erhalten. Besonders das
    Entwickeln von VR/AR Anwendungen konnte ich lernen und anwenden.\ Ich habe den Umgang mit MRTK gelernt und einblicke
    in das Implementieren von REST-Schnittstellen erhalten.\ Dazu habe ich auch das Entwickeln von Szenen in Unity
    lernen können und das Nutzen von Prefabs sowie den Umgang in Unity mit 3D-Objekten.
    Vor allem habe ich aber die implementierung eines HUDs gelernt und wie ich Daten mithilfe von REST und DjangoDB in
    einer Datenbank speichere und wieder abrufe.
    Außerdem habe ich kleinere Einblicke in das Leiten, Koordinieren und Planen eines Projektes erhalten.
    Einer meiner Hauptfehler am Anfang war es, denke ich am Anfang des Projektes aufgaben anbieten statt zu verteilen.
    So habe ich anfangs oft gefragt, wer welche Aufgaben übernehmen will.\ Dies sorgte oftmals zu Beginn, zu einer sehr
    unausgeglichenen Aufgabenverteilung.\ Gegen Mitte des Projekts bin ich dann dazu übergegangen, Aufgaben basierend auf
    aktuellem Workload und Präferenzen der Personen zu verteilen.\

\end{document}